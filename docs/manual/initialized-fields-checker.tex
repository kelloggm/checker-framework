\htmlhr
\chapterAndLabel{Initialized Fields Checker}{initialized-fields-checker}

The Initialized Fields Checker warns if a constructor does not initialize a
field.


\sectionAndLabel{Running the Initialized Fields Checker}{initialized-fields-checker-running}

An example invocation is

\begin{Verbatim}
javac -processor org.checkerframework.common.initializedfields.InitializedFieldsChecker MyFile.java
\end{Verbatim}

If you run it together with other checkers, then it issues warnings only if
the default value assigned by Java (0, false, or null) is not consistent
with the field's annotation, for the other checkers.
% It's actually the checkers and their subcheckers if any.  Saying that
% would be confusing to most users, who don't know what a "subchecker" is.
% The term "subchecker" appears only in the "creating a checker" section of
% the manual.
An example invocation is

\begin{Verbatim}
javac -processor ValueChecker,InitializedFieldsChecker MyFile.java
\end{Verbatim}


\sectionAndLabel{Motivation:  uninitialized fields}{initialized-fields-motivation}

Without the Initialized Fields Checker, every type system is
unsound with respect to fields that are never set.  (Exception:  The
Nullness Checker (\chapterpageref{nullness-checker}) is sound. Also, a type
system is sound if every annotation is consistent with 0, false, and null.)
Consider the following code:

\begin{Verbatim}
import org.checkerframework.checker.index.qual.Positive;

class MyClass {
  @Positive int x;
  MyClass() {
    // empty body
  }

  @Positive int getX() {
    return x;
  }
}
\end{Verbatim}

\noindent
Method \<getX> is incorrect because it returns 0, which is not positive.
However, the code type-checks because there is never an assignment to \<x>
whose right-hand side is not positive.
If you run the Index Checker together with the Initialized Fields Checker,
then the code correctly does not type-check.


\subsubsectionAndLabel{Remaining unsoundness}{initialized-fields-remaining-unsoundness}

Even with the Initialized Fields Checker, every type system (except the
Nullness Checker, \chapterpageref{nullness-checker}) is unsound with
respect to partially-initialized fields.  Consider the following code:

\begin{Verbatim}
import org.checkerframework.checker.index.qual.Positive;

class MyClass {
  @Positive int x;
  MyClass() {
    foo(this);
    x = 1;
  }

  @Positive int foo() {
    // ... use x, expecting it to be positive ...
  }
}
\end{Verbatim}

\noindent
Within method \<foo>, \<x> can have the value 0 even though the type of
\<x> is \<@Positive int>.


\sectionAndLabel{Example}{initialized-fields-example}

As an example, consider the following code:

\begin{Verbatim}
import org.checkerframework.checker.index.qual.Positive;

class MyClass {

  @Positive int x;
  @Positive int y;
  int z;

  // Warning: field y is not initialized
  MyClass() {
    x = 1;
  }
}
\end{Verbatim}

When run by itself, the Initialized Fields Checker warns that fields \<y>
and field \<z> are not set.

When run together with the Index Checker, the Initialized Fields Checker
warns that field \<y> is not set.  It does not warn about field \<z>,
because its default value (0) is consistent with its annotations.


\sectionAndLabel{Annotations}{initialized-fields-annotations}

The Initialized Fields type system uses the following type annotations:
\begin{description}
\item[\refqualclass{common/initializedfields/qual}{InitializedFields}]
  indicates which fields have definitely been initialized so far.
\item[\refqualclass{common/initializedfields/qual}{InitializedFieldsBottom}]
  is the type of \<null>.  Programmers rarely write this type.
\item[\refqualclass{common/initializedfields/qual}{PolyInitializedFields}]
  is a qualifier that is polymorphic over field initialization.
  For a description of qualifier polymorphism, see
  Section~\ref{method-qualifier-polymorphism}.
\end{description}

\begin{figure}
\includeimage{initializedfields}{4.5cm}
\caption{The type qualifier hierarchy of the Initialized Fields Checker.
\<@InitializedFieldsBottom> is rarely written by a programmer.}
\label{fig-initialized-fields-hierarchy}
\end{figure}

Figure~\ref{fig-initialized-fields-hierarchy} shows the subtyping
relationships among the type qualifiers.

There is also a method declaration annotation:

\begin{description}
\item[\refqualclass{common/initializedfields/qual}{EnsuresInitializedFields}]
  indicates which fields the method sets.  Use this for helper methods that
  are called from a constructor.
\end{description}


\sectionAndLabel{Comparison to the Initialization Checker}{initialized-fields-vs-initialization}

The Initialized Fields Checker is a lightweight version of the  Initialization Checker
(Section~\ref{initialization-checker}).  Here is a comparison between them.

\noindent
\begin{small}
\begin{tabular}{| l | l | l |}
 \hline
 & Initialization Checker & Initialized Fields Checker
 \\ \hline
 superclasses
 & tracks initialization of supertype fields
 & checks one class at a time
 \\
 partial initialization
 & changes the types of fields that are not initialized
 & unsound treatment of partially-initialized objects (*)
 \\
 type systems
 & works only with the Nullness Checker (**)
 & works for any type system
 \\
 disabling
 & always runs with the Nullness Checker
 & can be enabled/disabled per run
 \\
 \hline
\end{tabular}

\noindent
* See Section~\ref{initialized-fields-remaining-unsoundness} for an example.
\newline
** The Initialization Checker could be made to work with any type system, but
doing so would require changing the implementation of both the type system and
the Initialization Checker.
\end{small}


% LocalWords:  InitializedFields getX
% LocalWords:  InitializedFieldsBottom PolyInitializedFields
% LocalWords:  EnsuresInitializedFields
